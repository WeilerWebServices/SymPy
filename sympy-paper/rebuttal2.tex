\documentclass[answers,12pt]{exam}
\usepackage{xcolor}

\usepackage{xr-hyper}
\usepackage{hyperref}
\externaldocument[P-]{paper}
\externaldocument[S-]{supplement}

\usepackage{upquote}
\usepackage{textcomp}
\usepackage[T1]{fontenc}

\usepackage{pdfpages}

\hypersetup{
  colorlinks=true,
  filecolor=black, % color of cross-file links. Black because they only work
                   % if the paper sits alongside the rebuttal as "paper.pdf".
  urlcolor=blue % color of external links
}

\unframedsolutions
\shadedsolutions
\renewcommand{\solutiontitle}{}

\begin{document}

\includepdf{coverletter2.pdf}

\section{Comments from Editor}

\begin{questions}
\question  In section 2.3, lines 128--129, it is written
\begin{quote}
For instance, the identity $\sqrt{t^2} = t$ holds if $t$ is nonnegative ($t \geq
0$). However, for general complex $t$, no such identity holds.
\end{quote}
The first sentence doesn't make sense unless $\sqrt{\cdot}$ is defined;
obviously it is intended to yield the nonnegative square root. The second
sentence is incorrect, so needs rewording or deleting. For complex $t$ it is
true that $\sqrt{t^2} = t$ holds if $t$ lies in the right half-plane, assuming
$\sqrt{\cdot}$ is defined to be the square root lying in the right half-plane.
\begin{solution}
We reworked the two sentences to clarify the situation. We have added a
footnote with the definition of $\sqrt z$ that SymPy uses.
\end{solution}

\question Page 7, line 186: please state in the text whether this is typed as two lower case letter o's, as opposed to being some special symbol.
\begin{solution}
We have added a note here.
\end{solution}

\question Page 10, line 315: replace ``solving'' by ``finding its zeros''.
\begin{solution}
We have fixed this.
\end{solution}

\question Page 18, line 684: replace ``is as performant as'' by ``performs as well as''. The former is not correct English.
\begin{solution}
We have fixed this.
\end{solution}

\question Reference [42]: the title and publisher appear to have run together.
\begin{solution}
We have fixed this.
\end{solution}

\question Supplement, line 1: delete ``for''.
\begin{solution}
We have fixed this.
\end{solution}

\question Supplement, line 18: ``dependends'' $\rightarrow$ ``depends''.
\begin{solution}
We have fixed this.
\end{solution}

\end{questions}

\section{Comments from Reviewer 1}
\emph{There are no additional comments from reviewer 1.}


\section{Comments from Reviewer 2}
\subsection{Basic reporting}
No comments.

\subsection{Experimental design}
No comments.

\subsection{Validity of the findings}
No comments.

\subsection{Comments for the author}
This is a great paper. Thank you for addressing my concerns.

\section{Comments from Reviewer 3}
\subsection{Basic reporting}
The structure of the paper and the supplement is now improved, compared to the original submission, and the materials are now much easier to read and comprehend.

\subsection{Experimental design}
No Comments

\subsection{Validity of the findings}
No Comments

\subsection{Comments for the author}
\subsubsection{The paper}

\begin{questions}
\question The abstract and the paper itself mention the paper's supplement in several places. I believe it would be beneficial to add a short paragraph, possibly after line 90, that would list the supplement's contents in a manner similar to that of the paper's contents (lines 83--90), with the list submodules, to make it easier for an interested reader to find their way around it.
\begin{solution}
We have added a paragraph about the supplement at the suggested location. We
have also referenced specific supplement section numbers throughout the
paper.
\end{solution}

\question In line 323, it should say ``\ldots meaning that the entries\ldots''.
\begin{solution}
We have fixed this.
\end{solution}

\question In line 673, it should say either ``poorer than that of its commercial competitors'' or ``poorer than its commercial competitors''' (ending in apostrophe).
\begin{solution}
We have used the first suggestion.
\end{solution}

\question In line 813, one ``PhD thesis'' seems to be superficial.
\begin{solution}
We have fixed this.
\end{solution}

\subsubsection{The supplement}

\question Figure 1 is hard to read. I suggest putting the labels on the outside of the loops.
\begin{solution}
The figure code is generated automatically from the category theory submodule. We
have made this clearer by changing the figure caption to ``A diagram typeset
in Xy-pic automatically by \texttt{XypicDiagramDrawer}''. We have opened
\href{https://github.com/sympy/sympy/issues/11744}{issue 11744} in our public
issue tracker for improving the label placement.
\end{solution}

\question Figure 2 has the wrong link towards SymPy Gamma's computation of $\tan(x)$
 instead of $\int \tan(x)\,dx$.
\begin{solution}
The original link does include the integral steps, since SymPy Gamma
automatically includes the integral as one of the computations when given an
expression. However, we have changed the link as suggested to a query for
``integrate tan(x)'', as this is clearer.
\end{solution}

\end{questions}
\end{document}
