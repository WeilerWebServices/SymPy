SymPy calculates limits using the Gruntz algorithm, as described in%
~\cite{Gruntz1996limits}. The basic idea is as follows: any limit can be
converted to a limit $\lim\limits_{x\to\infty} f(x)$ by substitutions like
$x\to{1\over x}$. Then the subexpression $\omega$ (that converges
to zero as $x\to\infty$ faster than all other subexpressions) is identified in
$f(x)$, and $f(x)$ is expanded into a series with respect to $\omega$. Any
positive powers of $\omega$ converge to zero (while negative powers indicate
an infinite limit) and any constant term independent of
$\omega$ determines the limit. When a constant term still depends on
$x$ the Gruntz algorithm is applied again until a final numerical value
is obtained as the limit.

To determine the most rapidly varying subexpression, the comparability classes
must first be defined, by calculating $L$:
\begin{equation}
L\equiv \lim_{x\to\infty} {\log |f(x)| \over \log |g(x)|}
\end{equation}
The relations $<$, $>$, and $\sim$ are defined as follows: $f>g$ when
$L=\pm\infty$ (it is said that $f$ is more rapidly varying than $g$, i.e., $f$
goes to $\infty$ or $0$ faster than $g$), $f<g$ when $L=0$ ($f$ is less
rapidly varying than $g$) and $f\sim g$ when $L\neq 0,\pm\infty$ (both $f$ and
$g$ are bounded from above and below by suitable integral powers of the
other). Note that if $f > g$, then $f > g^n$ for any $n$. Here
are some examples of comparability classes:
\[2 < x < e^x < e^{x^2} < e^{e^x}\]
\[2\sim 3\sim -5\]
\[x\sim x^2\sim x^3\sim {1\over x}\sim x^m\sim -x\]
\[e^x\sim e^{-x}\sim e^{2x}\sim e^{x+e^{-x}}\]
\[f(x)\sim{1\over f(x)}\]

The Gruntz algorithm is now illustrated with the following example:
\begin{equation}
    \label{gruntz_example_fn}
f(x) = e^{x+2e^{-x}} - e^x + {1\over x} \,.
\end{equation}
First, the set of most rapidly varying subexpressions is determined---the
so-called \textit{mrv set}. For~\eqref{gruntz_example_fn}, the mrv set
$\{e^x, e^{-x}, e^{x+2e^{-x}}\}$ is obtained. These are all subexpressions of%
~\eqref{gruntz_example_fn} and they all belong to the same comparability
class. This calculation can be done using SymPy as follows:

% dict_keys output order varies
% no-doctest
\begin{verbatim}
>>> from sympy.series.gruntz import mrv
>>> mrv(exp(x+2*exp(-x))-exp(x) + 1/x, x)[0].keys()
dict_keys([exp(x + 2*exp(-x)), exp(x), exp(-x)])
\end{verbatim}

Next, an arbitrary item $\omega$ is taken from mrv set that converges to zero for
$x\to\infty$ and doesn't have subexpressions in the given mrv set.  If such a term is not
present in the mrv set (i.e., all terms converge to infinity instead of zero),
the relation $f(x)\sim {1\over f(x)}$ can be used.  In the
considered case, only the item $\omega=e^{-x}$ can be accepted.

The next step is to rewrite the mrv set in terms of $\omega=g(x)$.  Every element
$f(x)$ of the mrv set is rewritten as $A \omega^c$, where
\begin{equation}
\label{gruntz_rewrite}
c = \lim\limits_{x\to\infty} \frac{\log{f(x)}}{\log{g(x)}},
\qquad
A = e^{\log f - c \log g}
\end{equation}
Note that this step includes calculation of more simple limits, for instance
\begin{equation}
	\lim\limits_{x\to\infty} \frac{\log{e^{x + 2 e^{-x}}}}{\log e^{-x}}=
	\lim\limits_{x\to\infty} \frac{x + 2 e^{-x}}{-x} = -1
\end{equation}
In this example we obtain the rewritten mrv set: $\{{1\over\omega},
\omega, {1\over\omega}e^{2\omega}\}$. This can be done in SymPy with
\begin{verbatim}
>>> from sympy.series.gruntz import mrv, rewrite
>>> m = mrv(exp(x+2*exp(-x))-exp(x) + 1/x, x)
>>> w = Symbol('w')
>>> rewrite(m[1], m[0], x, w)[0]
1/x + exp(2*w)/w - 1/w
\end{verbatim}
Then the rewritten subexpressions are
substituted back into $f(x)$ in~\eqref{gruntz_example_fn}
and the result is expanded with respect to~$\omega$:
\begin{equation}
    \label{gruntz_example_fn2}
f(x) = {1\over x}-{1\over\omega}+{1\over\omega}e^{2\omega}
     = 2+{1\over x} + 2\omega + O(\omega^2)
\end{equation}

Since $\omega$ is from the mrv set, then in the limit as $x\to\infty$,
$\omega\to0$, and so $2\omega + O(\omega^2) \to 0$ in~\eqref{gruntz_example_fn2}:
\begin{equation}
f(x) = {1\over x}-{1\over\omega}+{1\over\omega}e^{2\omega}
    = 2+{1\over x} + 2\omega + O(\omega^2)
    \to 2 + {1\over x}
\end{equation}

In this example the result ($2+{1\over x}$) still depends on $x$, so the above procedure is
repeated until just a value independent of $x$ is obtained. This is the final
limit. In the above case the limit is $2$, as can be
verified by SymPy:\footnote{\label{suppnote:gruntz}To see the intermediate steps discussed above, interested
readers can switch on debugging output by setting
the environment variable \texttt{SYMPY\_DEBUG=True}, before importing anything from
the SymPy namespace.}

\begin{verbatim}
>>> limit(exp(x+2*exp(-x))-exp(x) + 1/x, x, oo)
2
\end{verbatim}

In general, when $f(x)$ is expanded in terms of $\omega$, the following is obtained:
\begin{equation}
f(x) = \underbrace{O\left({1\over \omega^3}\right)}_\infty
    + \underbrace{C_{-2}(x)\over \omega^2}_\infty
    + \underbrace{C_{-1}(x)\over \omega}_\infty
    + {C_{0}(x)}
    + \underbrace{C_{1}(x)\omega}_0
    + \underbrace{O(\omega^2)}_0
\end{equation}
The positive powers of $\omega$ are zero. If there are any negative powers of
$\omega$, then the result of the limit is infinity, otherwise the limit is
equal to $\lim\limits_{x\to\infty} C_0(x)$. The expression $C_0(x)$ is always
simpler than original $f(x)$, and the same is true for limits arising in the
rewrite stage~\eqref{gruntz_rewrite}, so the algorithm converges. A proof of this and further
details on the algorithm are given in Gruntz's PhD thesis~\cite{Gruntz1996limits}.
